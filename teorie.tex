\section*{Teoretická část}
Elementární buňka krystalu fluoridu lithného je plošně centrovaná a platí pro ní
\begin{equation}
\varrho = \frac{m}{V} = \frac{4m_u(A_{Li}+A_{F})}{(2d)^3} \,,
\end{equation}
kde $\varrho = \SI{2601}{\kg\per\metre\cubed}$ je hustota krystalu, $m_u$ je atomová hmotnostní konstanta, $d$ je hledaná vzdálenost hlavních atomových rovin a $A_X$ značí relativní atomovou hmotnost prvku $X$. Po úpravě a dosazení dostáváme
\begin{equation}
d=\sqrt[3]{\frac{m_u}{2\varrho}(A_{Li}+A_F)} \approx \SI{202.3}{\pico\metre} \,.
\end{equation}

Po dopadu na krystal se paprsek difraktuje. Dopadající i difraktovaný paprsek svírají s atomovými rovinami stejný úhel. Největší intenzitu má difraktovaný paprsek tehdy, když je splněna Braggova podmínka \cite{skripta}
\begin{equation}
\label{e:bragg}
2d \sin\vartheta = k\lambda \,,
\end{equation}
kde $\vartheta$ je úhel doplňkový k úhlu dopadu a $k$ je celé číslo.
Spektrum rentgenového záření proměříme tak, že změříme intenzitu difraktovaného paprsku při různých úhlech $\vartheta$, které odpovídají maximům prvního řádu různých vlnových délek.


Na krátkovlnné straně spektra bude od určité mezní hodnoty $\lambda_m$ již intenzita nulová. Platí \cite{skripta}
\begin{equation} \label{e:planck}
\frac{hc}{\lambda_m} = e U_a \,,
\end{equation}
kde $h$ je Planckova konstanta, $c$ je rychlost světla, $e$ je náboj elektronu a $U_a$ je anodové napětí rentgenky. Změřením $\lambda_m$ můžeme určit Planckovu konstantu.


V diskrétním spektru charakteristického záření budeme pozorovat dvě spektrální čáry $K_\alpha$ a $K_\beta$ odpovídající přechodu z energetické hladiny s hlavním kvantovým číslem $2 \to 1$ respektive $3 \to 1$.
Při přechodu z hladiny $n_1$ na $n_2$ se vyzáří foton s vlnočtem \cite{skripta}
\begin{equation} \label{e:cary}
\nu_{12}=R(Z-s)^2 \left( \frac{1}{n_2^2} - \frac{1}{n_1^2}  \right) \,,
\end{equation}
kde $R = \SI{1.0973e7}{\per\metre}$ je Rydbergova konstanta, $Z$ je atomové číslo prvku a $s$ je korekční stínící konstanta.
Spektrální čáře s vlnovou délkou $lambda$ odpovídá energetický rozdíl
\begin{equation}
E=\frac{hc}{\lambda} \,.
\end{equation}