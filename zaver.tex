\section*{Závěr}
Urřili jsme vzdálenost hlavních atomových rovin v krystalu fluoridu lithného $d = \SI{202.3}{\pico\metre}$.

Změřili jsme úhlovou závislost intenzity difraktovaného záření při pevné orientaci krystalu (viz graf \ref{g:uhel} a tabulka \ref{t:uhel}).

Dále jsme změřili spektrum rentgenového záření při anodovém napětí rentgenky $U_a = \SI{20}{\kV}$ (viz graf \ref{g:spektrum} a tabulka \ref{t:spektrum}).

Změřili jsme mezní vlnovou délku pro anodová napětí \SI{21}{\kV}, \SI{20}{\kV} a \SI{18}{\kV}, což nám umožnilo vypočítat Planckovu konstantu $\num{6.74(25)}\cdot \SI{e-34}{\joule\s}$.

Změřili jsme dvě spektrální čáry $K_\alpha$ a $K_\beta$ a určili jejich vlnovou délku, vlnočet, energetický rozdíl a stínící konstantu (viz \eqref{e:vysledky} výše).