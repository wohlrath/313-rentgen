\section*{Diskuze}
Úhlová závislost intenzity při pevné orientaci krystalu dopadla podle očekávání. Většina záření se difraktuje pod úhlem rovným úhlu dopadu.


Vlnové délky pozorovaných spektrálních čar se v rámci standardní odchylky neshodují s tabelovanými hodnotami $\lambda_{\alpha t} = \SI{154}{\pico\metre}$, $\lambda_{\beta t} = \SI{139}{\pico\metre}$, avšak v rámci $3\sigma$ už ano.

Námi změřená Planckova konstanta se shoduje s tabelovanou hodnotou \SI{6.626e-34}{\joule\s}.

Při určování mezních vlnových délek by bylo možné postupovat přesněji, kdybychom krátkovlnnou stranu spektra nafitovali přímkou a určili průsečík s nulou, místo toho abychom vzali nejkratší vlnovou délku, při které intenzita klesne pod určitou prahovou hodnotu. Stejně tak peaky od spektrálních čar by bylo možné nafitovat Gaussovou funkcí a jejich střed určit přesněji. Ještě větší přesnosti by se dalo dosáhnout přesnějším měřením úhlů a kratšími kroky v okolí těchto význačných vlnových délek. Proto jsme tento postup nepoužili.